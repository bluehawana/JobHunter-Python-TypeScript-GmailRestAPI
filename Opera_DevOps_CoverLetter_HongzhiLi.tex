\documentclass[a4paper,10pt]{letter}
\usepackage[left=1in,right=1in,top=1in,bottom=1in]{geometry}
\usepackage{enumitem}
\usepackage{titlesec}
\usepackage{hyperref}
\usepackage{graphicx}
\usepackage{xcolor}

% Define colors
\definecolor{darkblue}{rgb}{0.0, 0.2, 0.6}

% Section formatting
\titleformat{\section}{\large\bfseries\raggedright\color{black}}{}{0em}{}[\titlerule]
\titleformat{\subsection}[runin]{\bfseries}{}{0em}{}[:]

% Remove paragraph indentation
\setlength{\parindent}{0pt}

\begin{document}
\pagestyle{empty} % no page number

\begin{letter}{Opera\\V\"{a}stra Hamngatan 8\\411 17 G\"{o}teborg}

\opening{Hej Opera Team,}

\vspace{10pt}

I am writing to express my interest in the DevOps Engineer role. Beyond tooling and pipelines, my strongest value is operating as a cultural and functional bridge in a multi‑national enterprise. Working daily across Swedish and Chinese teams, I translate business goals into engineering roadmaps and turn engineering constraints into business‑ready decisions. This reduces friction, shortens feedback loops, and raises delivery quality through clearer expectations and shared vocabulary.

At ECARX, this “bridge” role shaped how we combined public cloud with private infrastructure. With a deep understanding of Azure Kubernetes Service, I helped design a pragmatic hybrid approach: AKS where it makes sense for elasticity and managed services; on‑prem HPC Kubernetes where cost/performance is critical. We balanced reliability and spend by instrumenting everything (Prometheus/Grafana/Loki/Alertmanager) and letting SLOs guide capacity choices. Around the edges, we used Cloudflare (CDN/WAF/R2), and selectively leveraged Heroku, Azure, AWS, and GCP to meet specific business demands while keeping operational complexity in check.

One concrete impact: together with our hardware engineer, visualization specialist, and DevOps team, we recorded a 7 minutes 47 seconds Android AOSP 15 clean build on our on‑prem server. That result, achieved through careful CPU/memory/I/O tuning, build cache strategy, and targeted parallelization, is competitive among the fastest publicly reported times and demonstrates both engineering depth and cross‑team coordination.

I am a lifelong learner (currently preparing the CNCF CKAD exam) and I enjoy bringing people together—product, platform, and application teams—to make complex systems feel simple and dependable. I would be excited to bring this mix of cultural fluency, hybrid‑cloud pragmatism, and delivery focus to Opera’s engineering organization in Gothenburg.

\vspace{12pt}

\closing{Sincerely,}

\signature{Hongzhi Li\\Ebbe Lieberathsgatan 27\\412 65 Göteborg\\hongzhili01@gmail.com\\0728384299\\2025.09.11}

\end{letter}

\end{document}
